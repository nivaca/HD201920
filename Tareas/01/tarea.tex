\documentclass[12pt,
              article,
              oneside
              ]{memoir}

\usepackage{ifluatex}

\ifluatex
  \usepackage{fontspec}
\fi

\usepackage[%english,
            %latin,
            spanish, es-minimal, es-nolists, es-nolayout, es-noshorthands, es-noquoting, es-uppernames, es-tabla
            ]{babel} % el último es el principal

\ifluatex
% LuaLaTeX --------------------
  \defaultfontfeatures{Ligatures=TeX, Scale=MatchLowercase}
  \setmainfont[Ligatures=Common,
    Numbers=OldStyle,
    SmallCapsFeatures = {LetterSpace=5.0},
    BoldFont = {LinLibertineOZ},
    % SmallCapsFeatures={Renderer=Basic},
    ]{Linux Libertine O}
  % \setsansfont{DejaVu Sans}
    \setmonofont[Scale=MatchLowercase]{DejaVu Sans Mono}
  % \setmonofont[Scale=MatchLowercase, Contextuals={Alternate}]{Fira Code}
  % \setmonofont[Scale=MatchLowercase, Contextuals={Alternate}]{DejaVu Sans Code}
\else
% PDFLaTeX ---------------------
  \usepackage[T1]{fontenc}
  \usepackage[utf8]{inputenc}
% \usepackage[osf]{coelacanth}
% \usepackage[osf]{XCharter}
% \usepackage[osf]{heuristica}
% \usepackage[p,osf]{baskervillef}
% \usepackage[osf]{cochineal}
% \usepackage{imfellEnglish}
% \usepackage[osf]{CormorantGaramond}
% \usepackage[oldstylenums, nott]{kpfonts}
% \usepackage[scaled=0.95]{inconsolata}
\usepackage[osf]{libertine}
% \usepackage[osf]{garamondx}
% \usepackage[osf]{Baskervaldx}
% \usepackage[osf]{newtxtext} % Times
% \usepackage{newtxmath} % Times
% \usepackage[osf]{newpxtext} % Palatino
% \usepackage{newpxmath} % Palatino
% \usepackage{cfr-lm} % Latin Modern with OSF
% \usepackage{lmodern} % Latin Modern
 \usepackage[scaled=0.8]{DejaVuSansMono}
\fi

\usepackage[babel=true, verbose=false, tracking=true, expansion=true, protrusion=true, final, draft=false]{microtype}
\SetTracking{ encoding = *, shape = sc }{ 30 }
\SetTracking[context = notracking, ]{encoding = *}{0}

\usepackage[autostyle=false, style=british]{csquotes}
% \usepackage[svgnames]{xcolor}
\usepackage{latexcolors}

% \usepackage[margin=3cm]{geometry}

% \usepackage{graphicx}

% \usepackage[
%   backend=biber,
%   style=authoryear-comp,
%   giveninits=true,
%   uniquename=init,
%   mincrossrefs=20,
% ]{biblatex}
% \addbibresource{}

% \hyphenpenalty=10000%
% \exhyphenpenalty=10000%
\righthyphenmin=62%
\lefthyphenmin=62%
\clubpenalty=9996
\widowpenalty=9999
\brokenpenalty=4991
\predisplaypenalty=10000
\postdisplaypenalty=1549
\displaywidowpenalty=1602
\flushbottom
\raggedbottom
\midsloppy
\parindent=0pt
\frenchspacing


\renewcommand\thesection{\arabic{section}}
\chapterstyle{reparticle}

\usepackage{enumitem}
\setlist{noitemsep, nosep}
\setlist[description]{font=\itshape}

\newcommand*{\elipsis}{~.~.~.}
\newcommand*{\elipsisb}{[~.~.~.~]}

\renewenvironment{quote}%
 {\begin{list}{}%
              {\setlength\rightmargin{0pt}%
              \setlength\leftmargin{\parindent}}%
  \item[]\small\ignorespaces}
 {\unskip\end{list}}

\setlength{\droptitle}{-2cm}
\title{\itshape *Title*}
\author{Nicolas Vaughan\\
  Universidad de los Andes, Bogotá\\
  \texttt{n.vaughan@uniandes.edu.co}%
  %\texttt{n.vaughan@oxon.org}%
  }
\date{}


\usepackage{hyperref}  % Last package to load

% ++++++++++++++++++++++++++++++++++++++++++++++++++++++++++
\begin{document}

\begin{center}
  \Large\bfseries
  Herramientas digitales 2019-20\\[6pt]

  \large
  Módulo: Texto digital\\[12pt]

\normalsize
  Profesor: Nicolás Vaughan

  (\href{mailto:n.vaughan@uniandes.edu.co}{n.vaughan@uniandes.edu.co})

\bigskip

Tarea No.~1:

Markdown, RegEx y Pandoc
\end{center}

\bigskip


\textbf{Fecha de entrega:} a más tardar el viernes 30 de agosto de 2019, 4:00 p.m.

\textbf{Modo de entrega:}Envíen a mi correo los dos archivos
resultantes (Markdown y HTML), con su apellido como
nombre de archivo (por ejemplo, \texttt{vaughan.md}, \texttt{vaughan.html}). El
asunto del correo debe ser \texttt{HD Tarea\ No.\ 1}.

\section{Descripción corta}

Deben procesar un archivo de texto plano, transformándolo con RegEx en
uno de formato Markdown, y finalmente en uno de formato HTML, con ayuda
de Pandoc.

\section{Descripción detallada}

Usando el motor RegEx del editor de texto Atom, deben procesar el texto
\emph{La gente cursi: Novela de Costumbres Ridículas}, de Ramón Ortega y
Frías (disponible aquí:
\url{http://bit.ly/2ZpPBEz}). Deben guardarlo
como un documento de Markdown (con extensión \texttt{.md}). (Eliminen
las primeras 62 líneas, y la parte en inglés al final, que no hacen
parte del texto.)

Procesen el texto con RegEx (buscando y reemplazando) para hacer lo
siguiente:

\begin{enumerate}

\item
  Unan las líneas (eliminando los saltos de línea) que componen los
  párrafos, teniendo cuidado de no eliminar los párrafos también. (Como
  vimos en la sesión virtual.)
\item
  Los dobles guiones (--) conviértanlos en rayas (---).\footnote{Las rayas son
  los guiones largos que se usan, entre otras, para indicar diálogos.
  Cf. \url{http://bit.ly/2MB0i19}}
\item
  Los comillones («») conviértanlos en comillas dobles (``'').\footnote{Deben ser comillas dos de apertura y cierre (``'') y no comillas dobles rectas ("").}
\item
  Eliminen los espacios en blanco dobles (triples, etc.) entre palabras
  y al principio de las líneas.
\item
  Reemplacen por asteriscos los guiones bajos que encierran algunas
  palabras (por ejemplo, de: \texttt{\_cursi\_} a: \texttt{*cursi*}).
\item
  Dejen los títulos de capítulos en \emph{una sola línea}. Por ejemplo,
  de:

\begin{verbatim}
CAPÍTULO I

La mujer casamentera.
\end{verbatim}

pasen a:

\begin{verbatim}
CAPÍTULO I: La mujer casamentera.
\end{verbatim}

Noten los dos puntos (:) luego del número romano.\footnote{Para este paso deberán
usar grupos de captura (\emph{capture groups}) de RegEx, para capturar
todo el texto desde ``Capítulo'' hasta el final de la línea. (Usen
paréntesis para eso en el campo de búsqueda, y luego \texttt{\$1} en el
campo de reemplazar. Vean http://bit.ly/322jFDI)}

\item
  Usando la sintaxis de Markdown, marquen los capítulos como encabezados
  (\emph{headers}) de nivel 2, como vimos en clase. (No se olviden del
  Epílogo y el Índice al final.) Antes de cada capítulo pongan una línea
  de separación (usando \texttt{***} en Markdown).
\item
  Marquen el título del libro y el autor como encabezados de nivel 1.
\item
  Limpien el índice del final del documento y denle formato de lista no
  numerada en Markdown.
\item
  Pongan los siguientes nombres en \textbf{negrita}: ``Robustiana'', ``Juanito'',
  ``Alfredo'', ``Bonacha'', ``Pascual'', ``Paquita'' y ``Adela''. (Esto
  se puede hacer esto en un solo paso, usando un grupo de captura
  alternado \texttt{(Alfredo\textbar{}Bonacha\textbar{}...)}.)
\item
  Finalmente, conviertan el documento a HTML usando Pandoc.
\end{enumerate}

\begin{center}\rule{0.5\linewidth}{\linethickness}\end{center}

A continuación encontrarán el texto resultante en PDF para que vean cómo
debe quedar. (Lo hice abriendo en Chrome el HTML y luego imprimiéndolo como PDF.)

\end{document}
